\documentclass[12pt,twoside]{article}
\usepackage{amsmath}
\date{\today}
\title{Notes on Transport Methods in General}
\author{Rachel Slaybaugh}
\begin{document}
\maketitle

\section{Quasi-Diffusion}

Miften and Larsen summary \cite{Miften1993}

Miften and Larsen re-derive the quasi-diffusion (QD) equations that has an "Eddington factor" that is invariante under a change in amplitude of $\psi$ and independent of $\psi$ when it is a linear function of $\mu$. The derivation they go through doesn't appear to hold is scattering is anything other than isotropic.

The ultimate contributions are
\begin{enumerate}
\item a derivaiton in spherical coordinates that guarantees positivity of the analytic solution
\item new diffusion boundary conditions in planar and spherical that are more accurate and efficient
\item discretization of the new analystic forms of the QD using short characteristics, which helps ensure positivity.
\end{enumerate} 

An interesting thing they do is apply a shift in the power iteration method for the eigenvalue. They subtract a perturbation of the smaller of the absoprtion cross section OR the fission term from each side. 

They also do a modified QD where they have an adjustable tuning parameter that's based on quadrature, $c$ and $\sigma_T h$ - turns out the Russians did this earlier (Anistratov comment). 

In general this is all just an acceleration method, so some of the math steps are pretty kludgey. 


\vspace*{2em}
The original work is by Gol'din \cite{Goldin1964}; it's old and I think the Miften and Larsen paper is much clearer. 

%------------- Bibliography --------------------
\bibliographystyle{model1-num-names}
\bibliography{te_general}

\end{document}