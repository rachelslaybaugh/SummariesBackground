\documentclass[12pt,twoside]{article}
\usepackage{amsmath}
\date{\today}
\title{Notes on Transport Methods in General}
\author{Rachel Slaybaugh}
\begin{document}
\maketitle

\section{Quasi-Diffusion}

Miften and Larsen summary \cite{Miften1993}

Miften and Larsen re-derive the quasi-diffusion (QD) equations that has an "Eddington factor" that is invariant under a change in amplitude of $\psi$ and independent of $\psi$ when it is a linear function of $\mu$. The derivation they go through doesn't appear to hold is scattering is anything other than isotropic.

The ultimate contributions are
\begin{enumerate}
\item a derivation in spherical coordinates that guarantees positivity of the analytic solution
\item new diffusion boundary conditions in planar and spherical that are more accurate and efficient
\item discretization of the new analytic forms of the QD using short characteristics, which helps ensure positivity.
\end{enumerate} 

An interesting thing they do is apply a shift in the power iteration method for the eigenvalue. They subtract a perturbation of the smaller of the absorption cross section OR the fission term from each side. 

They also do a modified QD where they have an adjustable tuning parameter that's based on quadrature, $c$ and $\sigma_T h$ - turns out the Russians did this earlier (Anistratov comment). 

In general this is all just an acceleration method, so some of the math steps are pretty kludgey. 


\vspace*{2em}
The original work is by Gol'din \cite{Goldin1964}; it's old and I think the Miften and Larsen paper is much clearer. 

\section{$\alpha$ eigenvalues}

What follows is from Hill~\cite{Hill1983} and \cite{Dahl2006}.\\
The time absorption, or $\alpha$, eigenvalue is an explicit eigenvalue problem. 
To go from time dependence to a time-independent eigenvalue problem, they assume time separability as
\[
\psi(\vec{r}, \vOmega, t) = \exp^{\alpha t) \psi(\vec{r}, \vOmega)
\]

The multigroup, $\alpha$-eval form of the TE is written as follows:
\[
\ve{L} \psi(\vec{r}, \vOmega) + (\ve{\Sigma_t} + \alpha \ve{V}^{-1}) \psi(\vec{r}, \vOmega) = (\ve{S} + \frac{1}{\lambda}\ve{F}) \psi(\vec{r}, \vOmega)
\]
here
\begin{align*}
\ve{L} &=\text{ leakage operator}\\
\ve{\Sigma_t} &=\text{ diagonal matrix of group total cross sections}\\
\ve{V} &=\text{ diagonal matrix of neutron group speeds}\\
\ve{S} &=\text{ scattering operator}\\
\ve{F} &=\text{ fission operator}\\
\lambda &=\text{ is an intermediate eigenvalue}
\end{align*}
We see a solution set $\alpha$, $\vec{\psi}$ that gives $\lambda = 1$. 
If $\alpha=0$, the $\lambda = k_{eff}$ .
Inner iterations are performed for a single group on the within-group scattering source.
Outer iterations are performed on the fission source, giving the intermediate eigenvalue.

The $\alpha$ eigenvalue problem is solved as a sequence of $\lambda$ eigenvalues until an $\alpha$ for which $\lambda = 1$ is found. 
This $\alpha$ iteration overlying the outer and inner iterations is
\[
\ve{L} \psi^k(\vec{r}, \vOmega) + (\ve{\Sigma_t} + \alpha^k \ve{V}^{-1}) \psi^k(\vec{r}, \vOmega) = (\ve{S} + \frac{1}{\lambda^k}\ve{F}) \psi^k(\vec{r}, \vOmega)
\]
The thing that varies between $\alpha$ eigenvalue methods is how you then choose $\alpha^{k+1}$ based on $\alpha^k$, $\lambda^k$, and $\psi^k$.


Next is one about using DSA \cite{Dahl2006}. The newest is this \cite{Fichtl}.


%------------- Bibliography --------------------
\bibliographystyle{siam}
\bibliography{te_general}

\end{document}