\documentclass[12pt,twoside]{article}
\usepackage{amsmath}
\date{\today}
\title{Notes on Transport Methods in General}
\author{Rachel Slaybaugh}

\newcommand{\vOmega}{\ensuremath{\hat{\Omega}}}
\newcommand{\ve}[1]{\ensuremath{\mathbf{#1}}}


\begin{document}
\maketitle

\section*{Quasi-Diffusion}

Miften and Larsen summary \cite{Miften1993}

Miften and Larsen re-derive the quasi-diffusion (QD) equations that has an "Eddington factor" that is invariant under a change in amplitude of $\psi$ and independent of $\psi$ when it is a linear function of $\mu$. The derivation they go through doesn't appear to hold is scattering is anything other than isotropic.

The ultimate contributions are
\begin{enumerate}
\item a derivation in spherical coordinates that guarantees positivity of the analytic solution
\item new diffusion boundary conditions in planar and spherical that are more accurate and efficient
\item discretization of the new analytic forms of the QD using short characteristics, which helps ensure positivity.
\end{enumerate} 

An interesting thing they do is apply a shift in the power iteration method for the eigenvalue. They subtract a perturbation of the smaller of the absorption cross section OR the fission term from each side. 

They also do a modified QD where they have an adjustable tuning parameter that's based on quadrature, $c$ and $\sigma_T h$ - turns out the Russians did this earlier (Anistratov comment). 

In general this is all just an acceleration method, so some of the math steps are pretty kludgey. 


\vspace*{2em}
The original work is by Gol'din \cite{Goldin1964}; it's old and I think the Miften and Larsen paper is much clearer. 

\section*{$\alpha$ eigenvalues}

What follows is from Hill~\cite{Hill1983}, \cite{Dahl2006}, and \cite{Shen2015} (piecing together the papers, I think I've got this right...).

The time absorption, or $\alpha$, eigenvalue is an explicit eigenvalue problem. 
We begin with the time dependent TE, assume a time separability for the flux, substitute that in to convert to the time independent $\alpha$-eval form of the TE, and simultaneously add a fictitious intermediate eigenvalue (note that the flux is a vector over energy; we are using the multigroup assumption):
\begin{align*}
\ve{V}^{-1} \frac{\partial \psi}{\partial t} &+ (\ve{L} + \ve{\Sigma_t}) \psi(\vec{r}, \vOmega, t) = (\ve{S} + \ve{F})\psi(\vec{r}, \vOmega, t)\\
\psi(\vec{r}, \vOmega, t) &= \exp^{\alpha t} \psi(\vec{r}, \vOmega)\\
%
\ve{L} \psi(\vec{r}, \vOmega) &+ (\ve{\Sigma_t} + \alpha \ve{V}^{-1}) \psi(\vec{r}, \vOmega) = (\ve{S} + \frac{1}{\lambda}\ve{F}) \psi(\vec{r}, \vOmega)
\end{align*}
%
here
\begin{align*}
\ve{L} &=\text{ leakage operator}\\
\ve{\Sigma_t} &=\text{ diagonal matrix of group total cross sections}\\
\ve{V} &=\text{ diagonal matrix of neutron group speeds}\\
\ve{S} &=\text{ scattering operator}\\
\ve{F} &=\text{ fission operator}\\
\lambda &=\text{ is an intermediate eigenvalue}
\end{align*}
We treat this like a $k$-eigenvalue problem, recognizing that $\alpha$ is unknown (if $\alpha=0$, then $\lambda = k_{eff}$). 

We search for a solution set, $\alpha$ and $\vec{\psi}$, that gives $\lambda = 1$.
Inner iterations are performed for a single group on the within-group scattering source.
Outer iterations are performed on the fission source, giving the intermediate eigenvalue.
We treat angle with a discrete ordinates approximation and space with something like a finite difference method. 

The $\alpha$ eigenvalue problem is solved as a sequence of $\lambda$ eigenvalue calculations until an $\alpha$ for which $\lambda = 1$ is found. 
We perform power iteration (or equivalent) to find each $\lambda$ value and then find $\alpha$. 
(Note, you cannot use PI to find $\alpha$ directly because the fundamental mode does not correspond to the largest eigenvalue).
The thing that varies between $\alpha$ eigenvalue methods is how you then choose $\alpha^{k+1}$ based on $\alpha^k$, $\lambda^k$, and $\psi^k$ using $\alpha$ iteration index $k$.

To find each $\lambda^k$, we do power iteration. 
In general, this requires a guess for the first $\alpha$, usually $\alpha=0$, and sometimes the second (depending on method). 
We also need a starting guess for $\lambda$ and $\psi$. Thus, we get an updated $\psi$ and $\lambda$ as:
\begin{align*}
\ve{L} \psi^k &+ (\ve{\Sigma_t} - \alpha^{k-1} \ve{V}^{-1}) \psi^k -\ve{S}\psi^k = \frac{1}{\lambda^{k-1}} \ve{F} \psi^{k-1}\\
\lambda^k &= \lambda^{k-1}\frac{\ve{F} \psi^k}{\ve{F} \psi^{k-1}}\:,
\end{align*}
which is done until 
\[
|\frac{\lambda^k}{\lambda^{k-1}} - 1 | \leq \epsilon^k\:.
\]

Now, we use the updated $\lambda$ to find $\alpha^k$ in hopes of getting a $\lambda^{k+1}$ that is closer to 1. 
This $\alpha$ iteration overlying the outer and inner iterations is
\[
\ve{L} \psi^k(\vec{r}, E, \vOmega) + (\ve{\Sigma_t} + \alpha^k \ve{V}^{-1}) \psi^k(\vec{r}, E, \vOmega) = (\ve{S} + \frac{1}{\lambda^k}\ve{F}) \psi^k(\vec{r}, E, \vOmega)
\]
The thing that varies between $\alpha$ eigenvalue methods is how you then choose $\alpha^{k+1}$ based on $\alpha^k$, $\lambda^k$, and $\psi^k$ or sometimes you use further past iterates. 
Lots of  papers on this.

Dahl talks about using DSA~\cite{Dahl2006}; Fichtl talks about a non-linear method~\cite{Fichtl}; Shen has a good lit review near the opening~\cite{Shen2015}.


%------------- Bibliography --------------------
\bibliographystyle{siam}
\bibliography{te_general}

\end{document}