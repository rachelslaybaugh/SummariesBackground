\documentclass[12pt,twoside]{article}
\usepackage{amsmath}
\usepackage{hyperref}

\newcommand{\ve}[1]{\ensuremath{\mathbf{#1}}}
\newcommand{\Sn}{\ensuremath{S_N}}
\newcommand{\denovo}{\textsf{Denovo}}
\newcommand{\kba}{\textsf{KBA}}

\newcommand{\vOmega}{\ensuremath{\ve{\Omega}}}
\newcommand{\hOmega}{\ensuremath{\hat{\ve{\Omega}}}}
\newcommand{\Ye}[2]{\ensuremath{Y^e_{#1}(\vOmega_#2)}}
\newcommand{\Yo}[2]{\ensuremath{Y^o_{#1}(\vOmega_#2)}}

\newcommand{\Jij}[2]{\ensuremath{\int L(\vOmega_#1)L(\vOmega_#2)}}

\newcommand{\sigg}[1]{\ensuremath{\sigma^{gg'}_{\text{s}\,#1}}}
\newcommand{\psig}{\ensuremath{\psi^g}}

\newcommand{\even}{\ensuremath{\phi^g}}
\newcommand{\odd}{\ensuremath{\vartheta^g}}

\newcommand{\evenp}{\ensuremath{\phi^{g'}}}
\newcommand{\oddp}{\ensuremath{\vartheta^{g'}}}

\newcommand{\qe}{\ensuremath{q^g}}
\newcommand{\qo}{\ensuremath{s^g}}

\newcommand{\apsi}[1]{\ensuremath{\psi^{\dagger\,#1}}}
\newcommand{\aeven}[1]{\ensuremath{\phi^{\dagger\,#1}}}
\newcommand{\aodd}[1]{\ensuremath{\vartheta^{\dagger\,#1}}}
\newcommand{\asigg}[1]{\ensuremath{\sigma^{g'g}_{\text{s}\,#1}}}

\newcommand{\Di}{\ensuremath{\Delta_i}}
\newcommand{\Dj}{\ensuremath{\Delta_j}}
\newcommand{\Dk}{\ensuremath{\Delta_k}}

\DeclareMathOperator{\octant}{Octant}

\date{\today}
\title{Notes on LDO implementation in Denovo}
\author{Rachel Slaybaugh}
%-----------------------------------------------------------
%-----------------------------------------------------------
\begin{document}
\maketitle

\section*{Operator Comparisons}
This is largely so I can get my brain wrapped around what is actually going where, mostly for sizing purposes and so I can figure out what we can reuse, etc.

%-----------------------------------------------------------
\subsection*{Original}
The traditional form of the equations are
\begin{align}
\ve{L}\vec{\psi} &= \ve{MS}\vec{\phi} + \vec{Q}\\
\vec{\phi} &= \ve{D}\vec{\psi}\:, \qquad \ve{D} = \ve{M}^T\ve{W} \\
\bigl(\ve{I} + \ve{DL^{-1} MS}\bigr)\vec{\phi} &= \ve{DL^{-1}}\vec{Q} 
\end{align}
%
More specifically, with the groups defined over the range $g\in[0,G]$,
at each spatial unknown we can write
\begin{multline}
    \ve{L}
    \begin{pmatrix}
      [\psi]_0 \\
      [\psi]_1 \\
      [\psi]_2 \\
      \vdots   \\
      [\psi]_G
    \end{pmatrix} =
    \begin{pmatrix}
      [\ve{M}]_{00} & 0 & 0 & 0 & 0 \\
      0 & [\ve{M}]_{11} & 0 & 0 & 0 \\
      0 & 0 & [\ve{M}]_{22} & 0 & 0 \\
      0 & 0 & 0 & \ddots & 0 \\
      0 & 0 & 0 & 0 & [\ve{M}]_{GG} \\
    \end{pmatrix}
    %%
    \begin{pmatrix}
      [\ve{S}]_{00} & [\ve{S}]_{01} & [\ve{S}]_{02} & \cdots &
      [\ve{S}]_{0G} \\
      [\ve{S}]_{10} & [\ve{S}]_{11} & [\ve{S}]_{12} & \cdots &
      [\ve{S}]_{1G} \\
      [\ve{S}]_{20} & [\ve{S}]_{21} & [\ve{S}]_{22} & \cdots &
      [\ve{S}]_{2G} \\
      \vdots & \vdots & \vdots & \ddots & \vdots \\
      [\ve{S}]_{G0} & [\ve{S}]_{G1} & [\ve{S}]_{G2} & \cdots &
      [\ve{S}]_{GG}
    \end{pmatrix}\\
    \times
    \begin{pmatrix}
      [\phi]_0 \\
      [\phi]_1 \\
      [\phi]_2 \\
      \vdots   \\
      [\phi]_G
    \end{pmatrix}
    %%
    +
    \begin{pmatrix}
      [\ve{M}]_{00} & 0 & 0 & 0 & 0 \\
      0 & [\ve{M}]_{11} & 0 & 0 & 0 \\
      0 & 0 & [\ve{M}]_{22} & 0 & 0 \\
      0 & 0 & 0 & \ddots & 0 \\
      0 & 0 & 0 & 0 & [\ve{M}]_{GG} \\
    \end{pmatrix}
    \begin{pmatrix}
      [q_e]_0 \\
      [q_e]_1 \\
      [q_e]_2 \\
      \vdots   \\
      [q_e]_G
    \end{pmatrix}\:,
    \label{eq:matrix-transport}
\end{multline}
where the notation $[\cdot]_g$ indicates a block matrix over all
unknowns for a single group.  Accordingly, we define
\begin{equation}
  [\psi]_g = \begin{pmatrix}
    \psi^g_1 & \psi^g_2 & \psi^g_3 & \cdots \psi^g_n
  \end{pmatrix}^T\:.
\end{equation}
%
The operator $\ve{M}$ is the moment-to-discrete matrix. It is used
to convert harmonic moments into discrete angles, and it is defined
\begin{equation}
  {\footnotesize
  [\ve{M}]_{gg} = \begin{pmatrix}
    \Ye{00}{1} & \Ye{10}{1} & \Yo{11}{1} & \Ye{11}{1} &
    \Ye{20}{1} & \cdots & \Yo{NN}{1} & \Ye{NN}{1} \\
    \Ye{00}{2} & \Ye{10}{2} & \Yo{11}{2} & \Ye{11}{2} &
    \Ye{20}{2} & \cdots & \Yo{NN}{2} & \Ye{NN}{2} \\
    \Ye{00}{3} & \Ye{10}{3} & \Yo{11}{3} & \Ye{11}{3} &
    \Ye{20}{3} & \cdots & \Yo{NN}{3} & \Ye{NN}{3} \\
    \vdots     & \vdots     & \vdots     & \vdots     &
    \vdots     &        & \vdots     & \vdots     \\
    \Ye{00}{n} & \Ye{10}{n} & \Yo{11}{n} & \Ye{11}{n} &
    \Ye{20}{n} & \cdots & \Yo{NN}{n} & \Ye{NN}{n}
  \end{pmatrix}}\:.
\end{equation}
%
The operator $\ve{D}$ is the discrete-to-moment matrix. It is used
to convert integrate discrete angles into harmonic moments, and it is defined
\begin{equation}
  {\footnotesize
  \ve{D} = \begin{pmatrix}
    w_1\Ye{00}{1} & w_2\Ye{00}{2} & w_3\Ye{00}{3} & \cdots & w_2\Ye{00}{n} \\ 
    w_1\Ye{10}{1} & w_2\Ye{10}{2} & w_3\Ye{10}{3} & \cdots & w_n\Ye{10}{n} \\
    w_1\Yo{11}{1} & w_2\Yo{11}{2} & w_3\Yo{11}{3} & \cdots & w_n\Yo{11}{n} \\
    w_1\Ye{11}{1} & w_2\Ye{11}{2} & w_3\Ye{11}{3} & \cdots & w_n\Ye{11}{n} \\
    w_1\Ye{20}{1} & w_2\Ye{20}{2} & w_3\Ye{20}{3} & \cdots & w_n\Ye{20}{n} \\
    \vdots        & \vdots        & \vdots        &        & \vdots        \\
    w_1\Yo{NN}{1} & w_2\Yo{NN}{2} & w_3\Yo{NN}{3} & \cdots & w_n\Yo{NN}{n} \\
    w_1\Ye{NN}{1} & w_2\Ye{NN}{2} & w_3\Ye{NN}{3} & \cdots & w_n\Ye{NN}{n}
  \end{pmatrix}}\:.
\end{equation}
%
The scattering cross sections are defined
\begin{equation}
  [\ve{S}]_{gg'} = \begin{pmatrix}
    \sigg{0} & 0 & 0 & 0 & 0 & 0 & 0 & 0 \\
    0 & \sigg{1} & 0 & 0 & 0 & 0 & 0 & 0 \\
    0 & 0 & \sigg{1} & 0 & 0 & 0 & 0 & 0 \\
    0 & 0 & 0 & \sigg{1} & 0 & 0 & 0 & 0 \\
    0 & 0 & 0 & 0 & \sigg{2} & 0 & 0 & 0 \\
    0 & 0 & 0 & 0 & 0 & \ddots   & 0 & 0 \\
    0 & 0 & 0 & 0 & 0 & 0 & \sigg{N} & 0 \\
    0 & 0 & 0 & 0 & 0 & 0 & 0 & \sigg{N}
  \end{pmatrix}\:.
\end{equation}

Finally, the flux and source moment vectors are defined
\begin{align}
  [\phi]_g &= \begin{pmatrix}
    \even_{00} & \even_{10} & \odd_{11} & \even_{11} & \even_{20}
    & \cdots & \odd_{NN} & \even_{NN}
  \end{pmatrix}^T\:,\\
  %%
  [q_e]_g &= \begin{pmatrix}
    \qe_{00} & \qe_{10} & \qo_{11} & \qe_{11} & \qe_{20}
    & \cdots & \qo_{NN} & \qe_{NN}
  \end{pmatrix}^T\:.
\end{align}


%-----------------------------------------------------------
\subsection*{LDO}
We are solving the \textit{LDO} equations, written in operator form in this way:
%
\begin{align}
\ve{L}\vec{\psi} &= \ve{J \Sigma_s}\vec{\psi} + \vec{Q}\\
\bigl(\ve{I} + \ve{L^{-1} J \Sigma_s}\bigr)\vec{\psi} &= \ve{L^{-1}}\vec{Q} \nonumber \\ 
\text{we can say }\ve{L^{-1}} &= \ve{I}\ve{L^{-1}} \qquad \text{and let } \ve{D} \equiv \ve{I} \text{, then} \nonumber\\
\bigl(\ve{I} + \ve{D L^{-1} J \Sigma_s}\bigr)\vec{\psi} &= \ve{D L^{-1}}\vec{Q}
\end{align}
% 
In these equations $\ve{J}$ contains the Lagrange expansion and $\ve{\Sigma_s}$ contains the scattering cross sections ($\Sigma_s(\hat{\Omega}_i \cdot \hat{\Omega}_j)$).
%
Now, 
\begin{multline}
    \ve{L}
    \begin{pmatrix}
      [\psi]_0 \\
      [\psi]_1 \\
      [\psi]_2 \\
      \vdots   \\
      [\psi]_G
    \end{pmatrix} =
    \begin{pmatrix}
      [\ve{J}] & 0 & 0 & 0 & 0 \\
      0 & [\ve{J}] & 0 & 0 & 0 \\
      0 & 0 & [\ve{J}] & 0 & 0 \\
      0 & 0 & 0 & \ddots & 0 \\
      0 & 0 & 0 & 0 & [\ve{J}] \\
    \end{pmatrix}
    %%
    \begin{pmatrix}
      [\ve{\Sigma_s}]_{00} & [\ve{\Sigma_s}]_{01} & [\ve{\Sigma_s}]_{02} & \cdots &
      [\ve{\Sigma_s}]_{0G} \\
      [\ve{\Sigma_s}]_{10} & [\ve{\Sigma_s}]_{11} & [\ve{\Sigma_s}]_{12} & \cdots &
      [\ve{\Sigma_s}]_{1G} \\
      [\ve{\Sigma_s}]_{20} & [\ve{\Sigma_s}]_{21} & [\ve{\Sigma_s}]_{22} & \cdots &
      [\ve{\Sigma_s}]_{2G} \\
      \vdots & \vdots & \vdots & \ddots & \vdots \\
      [\ve{\Sigma_s}]_{G0} & [\ve{\Sigma_s}]_{G1} & [\ve{\Sigma_s}]_{G2} & \cdots &
      [\ve{\Sigma_s}]_{GG}
    \end{pmatrix}\\
    \times
    \begin{pmatrix}
      [\psi]_0 \\
      [\psi]_1 \\
      [\psi]_2 \\
      \vdots   \\
      [\psi]_G
    \end{pmatrix}
    %%
    +
    \begin{pmatrix}
      [\ve{J}] & 0 & 0 & 0 & 0 \\
      0 & [\ve{J}] & 0 & 0 & 0 \\
      0 & 0 & [\ve{J}] & 0 & 0 \\
      0 & 0 & 0 & \ddots & 0 \\
      0 & 0 & 0 & 0 & [\ve{J}] \\
    \end{pmatrix}
    \begin{pmatrix}
      [q_e]_0 \\
      [q_e]_1 \\
      [q_e]_2 \\
      \vdots   \\
      [q_e]_G
    \end{pmatrix}\:,
    \label{eq:matrix-transport}
\end{multline}
where the notation $[\cdot]_g$ indicates a block matrix over all
unknowns for a single group.  Accordingly, we define
\begin{equation}
  [\psi]_g = \begin{pmatrix}
    \psi^g_1 & \psi^g_2 & \psi^g_3 & \cdots \psi^g_n
  \end{pmatrix}^T\:.
\end{equation}
%
The operator $\ve{M}$ is the moment-to-discrete matrix. It is used
to convert harmonic moments into discrete angles, and it is defined
\begin{equation}
  {\footnotesize
  [\ve{M}]_{gg} = \begin{pmatrix}
    \Ye{00}{1} & \Ye{10}{1} & \Yo{11}{1} & \Ye{11}{1} &
    \Ye{20}{1} & \cdots & \Yo{NN}{1} & \Ye{NN}{1} \\
    \Ye{00}{2} & \Ye{10}{2} & \Yo{11}{2} & \Ye{11}{2} &
    \Ye{20}{2} & \cdots & \Yo{NN}{2} & \Ye{NN}{2} \\
    \Ye{00}{3} & \Ye{10}{3} & \Yo{11}{3} & \Ye{11}{3} &
    \Ye{20}{3} & \cdots & \Yo{NN}{3} & \Ye{NN}{3} \\
    \vdots     & \vdots     & \vdots     & \vdots     &
    \vdots     &        & \vdots     & \vdots     \\
    \Ye{00}{n} & \Ye{10}{n} & \Yo{11}{n} & \Ye{11}{n} &
    \Ye{20}{n} & \cdots & \Yo{NN}{n} & \Ye{NN}{n}
  \end{pmatrix}}\:.
\end{equation}
%
The operator $\ve{D}$ is the discrete-to-moment matrix. It is used
to convert integrate discrete angles into harmonic moments, and it is defined
\begin{equation}
  {\footnotesize
  \ve{D} = \begin{pmatrix}
    w_1\Ye{00}{1} & w_2\Ye{00}{1} & w_3\Ye{00}{1} & \cdots & w_2\Ye{00}{1} \\ 
    w_1\Ye{10}{1} & w_2\Ye{10}{1} & w_3\Ye{10}{1} & \cdots & w_n\Ye{10}{1} \\
    w_1\Yo{11}{1} & w_2\Yo{11}{1} & w_3\Yo{11}{1} & \cdots & w_n\Yo{11}{1} \\
    w_1\Ye{11}{1} & w_2\Ye{11}{1} & w_3\Ye{11}{1} & \cdots & w_n\Ye{11}{1} \\
    w_1\Ye{20}{1} & w_2\Ye{20}{1} & w_3\Ye{20}{1} & \cdots & w_n\Ye{20}{1} \\
    \vdots        & \vdots        & \vdots        &        & \vdots        \\
    w_1\Yo{NN}{1} & w_2\Yo{NN}{1} & w_3\Yo{NN}{1} & \cdots & w_n\Yo{NN}{1} \\
    w_1\Ye{NN}{1} & w_2\Ye{NN}{1} & w_3\Ye{NN}{1} & \cdots & w_n\Ye{NN}{1}
  \end{pmatrix}}\:.
\end{equation}
%
The scattering cross sections are defined
\begin{equation}
  [\ve{S}]_{gg'} = \begin{pmatrix}
    \sigg{0} & 0 & 0 & 0 & 0 & 0 & 0 & 0 \\
    0 & \sigg{1} & 0 & 0 & 0 & 0 & 0 & 0 \\
    0 & 0 & \sigg{1} & 0 & 0 & 0 & 0 & 0 \\
    0 & 0 & 0 & \sigg{1} & 0 & 0 & 0 & 0 \\
    0 & 0 & 0 & 0 & \sigg{2} & 0 & 0 & 0 \\
    0 & 0 & 0 & 0 & 0 & \ddots   & 0 & 0 \\
    0 & 0 & 0 & 0 & 0 & 0 & \sigg{N} & 0 \\
    0 & 0 & 0 & 0 & 0 & 0 & 0 & \sigg{N}
  \end{pmatrix}\:.
\end{equation}

Finally, the flux and source moment vectors are defined
\begin{align}
  [q_e]_g &= \begin{pmatrix}
    \qe_{00} & \qe_{10} & \qo_{11} & \qe_{11} & \qe_{20}
    & \cdots & \qo_{NN} & \qe_{NN}
  \end{pmatrix}^T\:.
\end{align}

%-----------------------------------------------------------
\subsection*{Sizing}
The sizes used in Denovo are
\begin{equation}
  \begin{aligned}
    N_g &= \text{number of groups}\:,\\
    t   &= \text{number of moments}\:,\\
    n   &= \text{number of angles}\:,\\
    N   &= \text{$P_N$ order}\:,\\
    N_c &= \text{number of cells}\:,\\
    N_e &= \text{number of unknowns per cell}\:.
  \end{aligned}
\end{equation}
Now, we define
\begin{align}
  a &= N_g\times n\times N_c\times N_e\:,\\
  f &= N_g\times t\times N_c\times N_e\:.
\end{align}
In the new formulation I believe that $N$ becomes the Lagrange order, but that doesn't seem to factor in to any sizing.

The operator sizes of the \textbf{original} formulation are
\begin{align*}
\ve{I} &= (f \times f) \\
\ve{D} &= (f \times a) \\
\ve{L} = \ve{L^{-1}} &= (a \times a) \\
\ve{M} &= (a \times f) \\
\ve{S} &= (f \times f) \\
\vec{\phi} &= (f \times 1) \\
\vec{\psi} &= (a \times 1) \\
\vec{Q} &= (a \times 1)
\end{align*}

The operator sizes of the \textbf{new} formulation become
\begin{align*}
\ve{I} &= (a \times a) \\
\ve{D} &= (a \times a) \\
\ve{L} = \ve{L^{-1}} &= (a \times a) \\
\ve{M} &= (a \times a) \\
\ve{S} &= (a \times a) \\
\vec{\psi} &= (a \times 1) \\
\vec{Q} &= (a \times 1)
\end{align*}
I think this means that as long as we store the fluxes in the moments field (functionally such that $t$ = $n$?) then the sizing of everything will work out...


\end{document}