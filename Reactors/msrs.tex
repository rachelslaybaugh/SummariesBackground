\documentclass[12pt,twoside]{article}
\usepackage[letterpaper, textwidth=6.5in, textheight=9in]{geometry}
\usepackage{hyperref}
\usepackage{graphicx}
\usepackage{amssymb, amsthm, epsfig, amsmath}
\usepackage{mhchem}
\usepackage{paralist}
\usepackage{setspace}
\onehalfspacing

\newcommand{\ve}[1]{\ensuremath{\mathbf{#1}}}

\date{\today}
\title{Molten Salt Reactors}
\author{Rachel Slaybaugh}
%-----------------------------------------------------------
%-----------------------------------------------------------
\begin{document}
\maketitle

\section*{ORNL MSR Report}
\textbf{The Development Status of Molten-Salt Breeder Reactors}, ORNL report,
August 1972.

\subsection*{abstract}
\begin{compactitem}
\item MSR: high-temp, thermal-neutron breeder, Th-\ce{^{233}U} fuel cycle
\item continuous removal of \ce{Pa} and fission products
\item MSRE operated between 1965 and 1969
\end{compactitem}

\section*{Modeling MSRs with Serpent}
\textit{An extended version of the SERPENT-2 code to investigate fuel burn-up and core material evolution of the Molten Salt Fast Reactor}, Presented at the NuMat 2012 Conference, 22–25 October 2012, Osaka, Japan.

Need to deal with on-line fuel reprocessing, which prevents the use of commonly available burn-up codes. 



\end{document}
