\documentclass[12pt,twoside]{article}
\usepackage[letterpaper, textwidth=6.5in, textheight=9in]{geometry}
\usepackage{hyperref}
\usepackage{graphicx}
\usepackage{amssymb, amsthm, epsfig, amsmath}
\usepackage{mhchem}
\usepackage{paralist}
\usepackage{setspace}
\onehalfspacing

\newcommand{\ve}[1]{\ensuremath{\mathbf{#1}}}

\date{\today}
\title{Molten Salt Reactors}
\author{Rachel Slaybaugh}
%-----------------------------------------------------------
%-----------------------------------------------------------
\begin{document}
\maketitle

\section*{ORNL MSR Report}
\textbf{The Development Status of Molten-Salt Breeder Reactors}, ORNL report,
August 1972.

\subsection*{abstract}
\begin{compactitem}
\item MSR: high-temp, thermal-neutron breeder, Th-\ce{^{233}U} fuel cycle
\item continuous removal of \ce{Pa} and fission products
\item MSRE operated between 1965 and 1969
\end{compactitem}


%-------------------------------------------------------
%-------------------------------------------------------
\section*{Modeling MSRs with Serpent}
\textit{An extended version of the SERPENT-2 code to investigate fuel burn-up
and core material evolution of the Molten Salt Fast Reactor}, Presented at the
NuMat 2012 Conference, 22–25 October 2012, Osaka, Japan. (Journal of Nuclear 
Materials 441 (2013) 473--486)

\begin{compactitem}
\item Need to deal with on-line fuel reprocessing, which prevents the use of
commonly available burn-up codes. 
\item Table of current code capabilities; Serpent \textit{looks} best, but this
is a paper about Serpent
\item Common MSR parameters and those studied in the paper. Criticality is
maintained by extracting or injecting uranium.
\item Fig.\ 3 shows soluble vs.\ insoluble elements used in calculations
\end{compactitem}

In their calculations the total molar fraction of heavy metal fluorides is kept
constant. To do this, thee modify the Bateman equations in the isotopes present 
in the feed vector:
\[
\frac{dN_i}{dt} = \sum_j N_j \phi \cdot \sigma_{j\rightarrow i}  - \sum_j
N_j\phi \cdot \sigma_{i \rightarrow j} + \sum_j N_j \lambda_j b_{j\rightarrow
i} - N_i \lambda_i + c_i \sum_{k=HM} N_k\phi \cdot \sigma_{k,f}\:,
\]
where $c_i$ is the atomic fraction of $i$ in the feed vector and $\sigma_{k,f}$
is the one-group fission cross-section of the HM nuclide $k$.
They also have an equation for the removal of Li explicitly, and to remove
fission products they add the term
\[
-N_l \lambda_{l,repro}\:,
\]
where this is for fission product $l$.

"At each sub-step, the one-group cross sections
and neutron flux are interpolated between the two closest Monte
Carlo runs. The implementation of reactivity control in the SER-
PENT-2 depletion algorithm is briefly explained in the following.

At each sub-step, the one-group microscopic cross-sections,
along with the atomic densities provided by the solution of the system 
of burn-up equations, are adopted to calculate the infinite
multiplication factor $k_{inf}$ in the fuel material. In order to correctly
estimate the reactivity of the system, a correction for the leakage
term is needed. At each Monte Carlo run performed for neutron
spectrum calculation and one-group cross-section collapse, the
effective multiplication factor $k_{eff}$ is also estimated. At this point,
both $k_{eff}$ and $k_{inf}$ are known. A leakage correction factor can thus
be calculated at each Monte Carlo run. If this correction factor is
interpolated at each sub-step between the two closest calculations
(as it is done for one-group cross-sections), an accurate estimate of
the effective multiplication factor is available during the whole
simulation."

The algorithm they have implemented
\begin{enumerate}
\item For each burn-up step:
  \begin{enumerate}
  \item Estimate one-group cross-sections and $k_{eff}$ through a MC run with
  the initial material composition (predictor step)
  \item Calculate the leakage correction factor
  \end{enumerate}
\item For each sub-step:
  \begin{enumerate}
  \item Interpolate the one-group cross-sections and leakage correction factor
  between the two closest MC runs
  \item Solve the system of burn-up eqns.\ and calculate the isotopic
  composition at the end of the sub-step
  \item Estimate $k_{eff}$ through the one-group cross-sections and the leakage
  correction factor
  \item If $|k_{eff} - k_{eff,tgt}| > \epsilon$: adjust the fissile material
  injection/extraction rate and repeat the sub-step
  \end{enumerate}
\item Run an MC calculation with the final composition to complete the
corrector step; do the next burn-up step
\item Use the final composition as the initial composition for the next burn-up
step
\end{enumerate}

They found huge sensitivities to cross section libraries, but that was because
of U-233.


%-------------------------------------------------------
%-------------------------------------------------------
\section*{Francesco's Talk}
\textbf{Performance assessment of molten salt reactors fed with low enriched uranium}, Francesco Accardi, 28 Jan 2016.

\begin{compactitem}
\item Online reprocessing through continuous refueling
\item Need to take into account that power density is reduced by fuel being in primary loop
\item Basic parameter items; pros and cons of some materials
\item Fresh salt feed in, get rid of all noble metals, all gaseous fission products, and some salt (carrier salt + FPs + actinides)
\item MocDown extension for continuous Feed and Removal\textemdash may be of interest!
  \begin{compactitem}
  \item MocDown runs MCNP6.1 and ORIGEN2.2 to get equilibrium core composition
  \item new method modifies ORIGEN input by adding two new terns in the Bateman eqns.:
  \[\frac{dN_i}{dt} = \sum_j \bigl(f_{j\rightarrow i} \sigma_j^{tot} \phi + \gamma_{j\rightarrow i} \lambda_j\bigr)N_j - \bigl(\sigma_i^{tot}\phi + \lambda_i \bigr)N_i + F_i - RN_i\]
  \item Functionally, this deals with figuring out how much HM to add to get enough actinide moles and how much salt to remove to maintain the number of moles overall. This accounts for updating the appropriate number of fluorine atoms.
  \end{compactitem}
\end{compactitem}



\end{document}
